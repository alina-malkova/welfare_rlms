\begin{table}[htbp]
\centering
\caption{Partial Identification of Loss Aversion Parameter}
\label{tab:lambda_bounds}
\begin{tabular}{lcccc}
\toprule
 & \multicolumn{2}{c}{Point Estimate} & \multicolumn{2}{c}{95\% CI} \\
Curvature ($\eta$) & $\lambda$ (Formal) & $\lambda$ (Informal) & Lower & Upper \\
\midrule
0.50 & 1.099 & 3.877 & 1.933 & 6.491 \\
0.75 & 1.065 & 2.468 & 1.552 & 3.480 \\
0.88 & 1.055 & 2.160 & 1.454 & 2.894 \\
1.00 & 1.048 & 1.969 & 1.390 & 2.548 \\
\midrule
\multicolumn{5}{l}{\textit{Identified set over $\eta \in [0.1, 1.0]$:}} \\
 & & [2.041, 876.433] & [1.415 &  1.2e+04] \\
\bottomrule
\end{tabular}
\begin{tablenotes}
\small
\item Notes: Loss aversion parameter $\lambda$ identified from asymmetric 
consumption responses to positive vs negative income shocks. 
 = |\beta^-| / |\beta^+|$ is the ratio of responses. 
Under prospect theory, $\lambda(\eta) = R^{1/\eta}$. 
Kahneman-Tversky benchmark: $\eta = 0.88$, $\lambda = 2.25$.
\end{tablenotes}
\end{table}
